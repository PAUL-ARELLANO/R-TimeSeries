\documentclass{article}
\usepackage{natbib}
\usepackage{fullpage}
%\usepackage{neurips_2024}


% to compile a preprint version, e.g., for submission to arXiv, add add the
% [preprint] option:
%     \usepackage[preprint]{neurips_2024}


% to compile a camera-ready version, add the [final] option, e.g.:
%     \usepackage[final]{neurips_2024}


% to avoid loading the natbib package, add option nonatbib:
%    \usepackage[nonatbib]{neurips_2024}

\usepackage{tikz}
\usepackage[utf8]{inputenc} % allow utf-8 input
\usepackage[T1]{fontenc}    % use 8-bit T1 fonts
\usepackage{hyperref}       % hyperlinks
\usepackage{url}            % simple URL typesetting
\usepackage{booktabs}       % professional-quality tables
\usepackage{amsfonts}       % blackboard math symbols
\usepackage{amsmath}
\usepackage{nicefrac}       % compact symbols for 1/2, etc.
\usepackage{microtype}      % microtypography
\usepackage{xcolor}         % colors
\usepackage{graphicx}

\title{Change Detection Modelling across Time Series}

\author{%
  Paul Arellano --- \texttt{Paul.Arellano@nau.edu}\\
  Alexander F Shenkin --- \texttt{Alexander.Shenkin@nau.edu}\\
  ... --- \texttt{emailaddress@nau.edu}
}

\begin{document}

\maketitle

\begin{abstract}
To be prepared...
...
...
...

\end{abstract}

\section{Introduction}
\label{sec:intro}
This document explains a process to apply several change detection models in time series of vegetation indexes to detect tree stress caused by beetle infestation and drought hit. This study relies on two time series datasets: The first data set consists of 1,300 synthetic daily time series of NDVI for the years 2019-2025 created by a fully factororial experimental design that uses a combination of sine functions to simulate seasonal variations in NDVI values, with added noise, variations, linear trends, and random noise. The bark beetle event is simulated as a progressive decline in NDVI starting from a specific date; and the second data set consists of ## time series retrieved from PLANET satellite images at 3 meter resolution from forest sites identified by the USFS as stressed forests caused by insects infestation and drought events. Later, several change detection models were applied to those datasets to detect changes. We assessed the performance of each model  








\section{Datasets and Field sites}
\paragraph{Description of the synthetic time series datasets}
Details on how the synthetic datasets were created and full details of the results throughout statistical methods


\paragraph{Description of PLANET time series datasets}



\paragraph{Study sites description} 



 
\section{Change detection models}














\section{Related work}
\label{sec:related-work}

\section{Results}
\label{sec:methods}

In this section ...
\subsection{Change detection on Synthetic datasets}

\subsection{Change detection on PLANET time series}


\section{Concusions}
\label{sec:results}
In this section, 



\bibliographystyle{abbrvnat}
\bibliography{refs}


\end{document}